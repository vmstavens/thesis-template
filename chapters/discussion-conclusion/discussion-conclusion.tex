\chapter{Discussion \& Conclusion}\label{ch:discussion-conlcusion}
Composing the discussion and conclusion chapters of an engineering thesis requires a comprehensive synthesis of the research findings and their broader implications. In the discussion chapter, delve into the results, critically analyzing them within the context of the research objectives and hypotheses. Highlight patterns, trends, and deviations, offering possible explanations supported by literature. Address any limitations or uncertainties and explore potential avenues for future research. Transitioning to the conclusion chapter, recap the main contributions of the study, reaffirming how they align with the initial research questions. Summarize the significance of the findings in the broader field of engineering and their real-world applications. A strong conclusion should reflect on the relevance of the research and its potential impact on the industry or society. Emphasize the implications of the study's outcomes, showcasing how they fill gaps in knowledge, address challenges, or advance technology. Ultimately, these chapters serve as a platform to demonstrate your grasp of the subject matter, your ability to critically evaluate outcomes, and your aptitude for drawing valuable insights from complex data, all of which underpin the holistic contribution of your engineering thesis to the field.