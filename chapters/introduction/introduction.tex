\chapter{Introduction}\label{ch:intro}

\section{Context}\label{sec:intro-context}

The contextual section within a thesis introduction serves as a bridge between the general knowledge of the field and the specific focus of your research. This section provides readers with essential background information that helps them understand the broader context within which your study operates. To write an effective contextual section, begin by outlining the foundational concepts, theories, and existing research relevant to your topic. Highlight key developments, debates, and gaps in the literature that your research aims to address. You can also mention any real-world implications or applications of your work. By carefully weaving together the established knowledge in the field and your research's niche, the contextual section sets the stage for the reader, preparing them to appreciate the significance and uniqueness of your study. Balancing conciseness with clarity, the section should smoothly transition from the general to the specific, ensuring that your research's importance is clearly conveyed within the broader academic landscape.

\section{Problem Description}\label{sec:intro-problem-description}
Writing a problem description in a thesis involves articulating the central challenge or question that your research aims to address. It is a critical section that sets the context for your work, highlighting its significance and relevance. To effectively write a problem description, start by clearly defining the problem, outlining its scope, and emphasizing its real-world implications. Provide background information to help readers understand the context and the gaps in existing knowledge or practices that your research seeks to fill. Use clear and concise language to convey the problem's complexity while avoiding unnecessary jargon. Consider incorporating relevant statistics, anecdotes, or examples to illustrate the problem's impact. Additionally, acknowledge existing research and solutions related to the problem, emphasizing the unique perspective or approach your study brings. Ultimately, a well-crafted problem description should engage readers' interest and lay the foundation for the subsequent sections of your thesis. \medskip

The problem can ideally be decomposed into some number of sub-problems, that here can be presented. These can later have their own chapters, literature review etc.
% and end effector pose est
\section{Thesis Overview}\label{sec:intro-thesis-overview}

Composing a thesis overview is a foundational step in guiding readers through the content and scope of your research. This succinct yet informative section serves as a roadmap, providing readers with a clear understanding of the purpose, structure, and key elements of your thesis. Begin by introducing the broader research topic and its significance, highlighting the gap in existing knowledge or the problem you aim to address. Subsequently, briefly outline the main research questions, hypotheses, or objectives that your thesis seeks to answer or achieve. Mention the methodology or approach you adopted and provide a glimpse of the primary findings or outcomes. Additionally, touch upon the organization of the subsequent chapters, delineating how each chapter contributes to the overall narrative. The overview should strike a balance between providing enough context for readers to engage with your work and maintaining conciseness to maintain their interest. In essence, a well-crafted thesis overview offers a panoramic view of your research journey, setting the stage for a coherent and engaging exploration of your thesis.