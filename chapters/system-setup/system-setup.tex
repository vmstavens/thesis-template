\chapter{System Setup}\label{ch:system-setup}
Writing a comprehensive system setup section in an engineering thesis is pivotal for providing readers with a clear understanding of the technical infrastructure and experimental setup employed in your research. This section acts as a roadmap, guiding readers through the essential components and configurations of the system. Begin by detailing the hardware and software components, including specific models, versions, and specifications, as these are crucial for reproducibility. Explain the rationale behind your choices and how they align with your research objectives. Diagrams, flowcharts, and illustrations can greatly enhance comprehension. Address any calibration procedures, safety measures, or environmental conditions that impact the system's functionality. Additionally, highlight any customizations or modifications made to existing systems. A well-written system setup not only demonstrates your mastery of the technical aspects but also lays the foundation for readers to grasp the context and validity of your experimental results. \medskip

Make sure to make it clear what parts are provided and what are developed. This holds for both software and hardware. A graph or some data to demonstrate the provided systems validity can be used with benefit.